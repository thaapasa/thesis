%%%%%%%%%%%%%%%%%%%%%%%%%%%%%%%%%%%%%%%%%%%%%%%%%%%%%%%%%%%%%%%%%%%%
%-------------------------------------------------------------------
% Introduction
%-------------------------------------------------------------------
%%%%%%%%%%%%%%%%%%%%%%%%%%%%%%%%%%%%%%%%%%%%%%%%%%%%%%%%%%%%%%%%%%%%
\selectlanguage{british}
\chapter{Introduction}
\label{toc:intro}

As the software engineering field develops, traditional approaches to 
software development start to be inadequate. Customers of software 
development companies have more and more complicated requirements for 
their software. It is no longer feasible to develop all the components 
of a system in-house \citep{usingj2ee}. Customers expect the software 
to conform to their needs, even though they are not always able to 
define them exactly beforehand. Traditional software development 
processes, in which the customer's first touch to the developed 
software is in the end of the project, start falling apart. To remedy 
these problems, new approaches are required \citep{agileinnovation}.

Fortunately many software libraries, frameworks and tools exist on the 
market to aid the software development process. Traditionally these 
tools and libraries have been proprietary, but more and more extensive 
and robust solutions are starting to appear on the market from the 
open source community. Open-source software is software that can be 
freely used, modified and distributed provided that certain 
restrictions are observed \citep{usingoss}. However, for most 
purposes, the majority of open-source solutions can be used freely, 
even in a commercial project. \cite{ossolutions} provides numerous 
proficient arguments for using open-source software in the commercial 
world, such as low costs, no vendor lock-in, and reliability. Open 
source projects are guided by technology instead of business needs, 
with technical excellence often as the primary goal.

A traditional software development process identifies change as the 
enemy, and sets out to minimise the amount of change required 
throughout the project. This has, however, proven to be a badly 
working approach \citep{agileinnovation}. The requirements set out at 
the beginning of the project are often not final, but change as the 
project progresses. The initial requirements may be incorrect, even if 
no apparent change has happened during the project. Presently, 
eliminating change early means being unable to meet changing business 
conditions, which leads to business failure \citep{agileinnovation}. 
Agile software development methods take an alternative viewpoint to 
this problem: they embrace change. Instead of trying to plan 
everything at the beginning of a project, the agile approach is to 
support change. The agile practices and methods are designed to 
diminish the cost of change, and to provide early feedback to the 
customer. \cite{agilesdm} provide an excellent summary of modern agile 
methodologies.


%%%%%%%%%%%%%%%%%%%%%%%%%%%%%%%%%%%%%%%%%%%%%%%%%%%%%%%%%%%%%%%%%%%%
% Scope
%%%%%%%%%%%%%%%%%%%%%%%%%%%%%%%%%%%%%%%%%%%%%%%%%%%%%%%%%%%%%%%%%%%%
\section{Scope}
\label{toc:intro:scope}

Both open-source software and agile methodologies are starting to gain 
grounds in the literature. Studies favouring the adoption of agile 
practices and open-source solutions have begun to emerge 
\citep{agileadoption,usingoss}. However, studies concerning the usage 
of open-source solutions in agile development are scarce. This thesis 
sets out to study and test \textsl{how well a set of mature 
open-source solutions can be used in a commercial project that is 
managed using agile methodologies}. This is evaluated by initiating 
and analysing a case study project.

To accomplish this goal, this thesis studies the current state of 
software development from the viewpoint of modern Java web 
applications. The Java programming language provides a solid, robust 
base for enterprise applications, and Java was a requirement of the 
case study project. Web applications, on the other hand, are an 
interesting test subject, because they can be developed rapidly, yet 
larger web applications require a full-blown server architecture that 
needs to be comprehensive, scalable and distributable. Additionally, 
web applications are often usable from anywhere in the world, so they 
potentially have a very large user group. This thesis concentrates on 
discovering the areas of web development in which libraries, tools and 
frameworks can be used. Consequently, software solutions from the open 
source market are presented to match the requirements. The selection 
criteria for the open-source solutions is that they must be reliable, 
widely-used mainstream products with a large user base.

Furthermore, some agile development methods are introduced and 
studied. In this thesis, the practices of two most popular agile 
methodologies, Extreme Programming \citep{xpexplained} and Scrum 
\citep{scrumprocess} are studied in detail. Extreme Programming 
describes a software development approach that consists of a number of 
common-sense development practices that together form an efficient, 
yet light-weight process. \abbrev{XP} relies on simple initial design 
and supports even large restructurings later on. Scrum, on the other 
hand, concentrates on agile project management and not on the used 
development methods. Again, these were selected because they are 
presently two of the most popular agile methodologies 
\citep{rapidxp,agilesdm} and they can be used together because they 
target different aspects of the software development process. 
Additionally, the adoption of agile methodologies is discussed. 
\cite{agileadoption} reports that it is better to select only a few 
practices when adopting an agile process instead of trying to tackle 
everything at once.

To test the selected approach, a commercial case study project is 
carried out using the presented open-source solutions and a subset of 
the described agile practices. In the project, \finnish{HiQ Softplan 
Oy} develops a real estate transaction system for \finnish{GVA Finland 
Oy}. The resulting software is measured and evaluated with the 
\abbrev{CK} metric suite \citep{oodmetrics}, and the adoption of 
\abbrev{XP} practices is evaluated by using the \abbrev{XP} Evaluation 
Framework \citep{xpevaluationfw}. Finally, the usage of open-source 
software in agile development projects is analysed based on both the 
results obtained from these metrics and subjective evaluation of the 
project.


%%%%%%%%%%%%%%%%%%%%%%%%%%%%%%%%%%%%%%%%%%%%%%%%%%%%%%%%%%%%%%%%%%%%
% Outline of the Work
%%%%%%%%%%%%%%%%%%%%%%%%%%%%%%%%%%%%%%%%%%%%%%%%%%%%%%%%%%%%%%%%%%%%
\section{Outline of the Work}
\label{toc:intro:outline}

The structure of this thesis follows the layout described in the 
previous section closely. First, chapter \ref{toc:webdevel} describes 
the current state of web software development and presents several 
areas in which existing, reusable software solutions could be used. 
Subsequently, in chapter \ref{toc:oss}, software solutions from the 
open source market are identified to meet these requirements. After 
that, chapter \ref{toc:agile} introduces the relatively new agile 
methodologies and describes the practices of \abbrev{XP} and Scrum. 
Chapter \ref{toc:evaluation} presents metrics for measuring software 
quality and evaluating the adoption of agile practices. Chapters 
\ref{toc:project} and \ref{toc:selected} describe the project 
requirements and the selected approach for carrying out the project in 
detail. After that, chapter \ref{toc:result} goes through the results 
obtained from the selected metrics and the evaluation of the project. 
Finally, chapter \ref{toc:conclusions} presents the conclusions based 
on the executed project.


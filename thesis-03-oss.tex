%%%%%%%%%%%%%%%%%%%%%%%%%%%%%%%%%%%%%%%%%%%%%%%%%%%%%%%%%%%%%%%%%%%%
%-------------------------------------------------------------------
% Open-Source Web Development Solutions
%-------------------------------------------------------------------
%%%%%%%%%%%%%%%%%%%%%%%%%%%%%%%%%%%%%%%%%%%%%%%%%%%%%%%%%%%%%%%%%%%%
\selectlanguage{british}
\chapter{Open-Source Web Development Solutions}
\label{toc:oss}

The previous chapter identified several areas of interest in web 
development. In this chapter open-source solutions for those areas are 
presented. The chapter begins with an introduction to open-source 
software. After that, the usage of open-source software is justified. 
Finally, the selected open-source solutions are presented.


%%%%%%%%%%%%%%%%%%%%%%%%%%%%%%%%%%%%%%%%%%%%%%%%%%%%%%%%%%%%%%%%%%%%
% Introduction to Open Source
%%%%%%%%%%%%%%%%%%%%%%%%%%%%%%%%%%%%%%%%%%%%%%%%%%%%%%%%%%%%%%%%%%%%
\section{Introduction to Open Source}
\label{toc:oss:intro}

Open-source\footnote{\citep{osi} suggests that when using the term
\textsl{open source} as an adjective, it should be hyphenated.} 
software, in contrast to proprietary software, is software that must 
have its source code available for developers to see, improve and to 
create derivative works from. Open source is not shareware, which is 
mostly inexpensive proprietary software that can be tested freely 
before buying it. On the other hand, open-source software does not 
have to be free, in the meaning that users would not have to pay for 
obtaining it. The defining characteristics of open-source software are 
that users must be able to freely use, redistribute, modify and 
distribute derivatives from the work, under the terms of the original 
license \citep{ossolutions}. There are other requirements for 
open-source software, but those are not covered in this work.

The full definition of \abbrev{OSS} is that it must meet the Open 
Source Definition (\abbrev{OSD}), which is maintained by the Open 
Source Initiative \citep{osi}. \abbrev{OSI} is a non-profit 
corporation that promotes the usage of open-source software in the 
business world. In theory, anyone can contribute to the development of 
an open-source program. This is a radical change from the proprietary, 
in-house software development style. In the \abbrev{OS} development 
model the software that is being created can be inspected by thousands 
of developers from all around the world, instead of the selected few 
developers of a proprietary software. According to \cite{osi}, this 
reduces the development time of the software and improves the quality. 
When the software is exposed, the software defects can also be found 
faster.

For developing software for your own use, such as creating a web 
application that is installed on one of your own servers, there are 
usually no restrictions on the use of open-source software. 
\abbrev{OSS} licences most often only apply when you are distributing 
the software. For example, the common \abbrev{GPL} license places 
requirements for you only if you create a software derived from or 
added to a \abbrev{GPL} software that you \textsl{actually distribute 
to someone} \citep{ossolutions}. Nevertheless, the license terms must 
be examined carefully when taking open-source solutions into use 
\citep{usingoss}.


%%%%%%%%%%%%%%%%%%%%%%%%%%%%%%%%%%%%%%%%%%%%%%%%%%%%%%%%%%%%%%%%%%%%
% Using Open-Source Software
%%%%%%%%%%%%%%%%%%%%%%%%%%%%%%%%%%%%%%%%%%%%%%%%%%%%%%%%%%%%%%%%%%%%
\section{Using Open-Source Software}
\label{toc:oss:usage}

Open-source software is generally surrounded by myths and beliefs that 
it cannot really compete with proprietary software. These beliefs are 
referred to by open source proponents as \abbrev{FUD} for the fear, 
uncertainty and doubt that they generate \citep{ossolutions}. It is 
feared that the quality of the open-source software is not high 
enough, that it is not documented, not standard and not mature enough. 
Another fear is that the development and maintenance of a selected 
open-source software stops, and soon the software cannot be used 
anymore because of unfixed bugs or new technologies that are not 
supported by the software.

Unfortunately, new Internet technologies are often buggy and 
non-standard, and this certainly applies to open-source software as 
well \citep{usingj2ee}. However, there is one point about the quality 
of open-source software that is not often thought of: the \abbrev{OSS} 
development process is not tied to deadlines and budgets that haunt 
the proprietary software development process. Instead, \abbrev{OSS} 
development is driven by the technology itself. Often the primary goal 
of an \abbrev{OSS} development project is technical excellence 
\citep{ossolutions}. In comparison, proprietary software always has 
business interests at the background, and sometimes those business 
interests drive unfinished software to the market to meet the 
deadlines.

While it is possible for a very small open source project to die and 
leave its users stranded with a piece of software they cannot use, a 
mainstream open-source software that has a large user community is a 
lasting solution that can be safely taken into use \citep{usingoss}. 
If developers leave the \abbrev{OSS} development project others will 
take their place.

Many people fear that taking an open-source software into use 
restricts the project into using only open-source software 
\citep{ossolutions}. In truth, almost everyone is already using 
open-source software, perhaps without even realising it. The 
best-known open-source software is the Linux operating system, which 
has surpassed the various proprietary Unix operating systems in the 
server operating system market \citep{exploratoryos}. Another 
well-known \abbrev{OS} software, the open-source Apache Web server 
\citep{apache}, is used in over 60\% of the current web server 
installations \citep{websurvey}.


%%%%%%%%%%%%%%%%%%%%%%%%%%%%%%%%%%%%%%%%%%%%%%%%%%%%%%%%%%%%%%%%%%%%
% Open-Source Solutions
%%%%%%%%%%%%%%%%%%%%%%%%%%%%%%%%%%%%%%%%%%%%%%%%%%%%%%%%%%%%%%%%%%%%
\section{Open-Source Solutions}
\label{toc:oss:selected}

This section presents solutions for the different areas in web 
development that were identified in chapter~\ref{toc:webdevel}. In 
addition, some \abbrev{OS} solutions for other development supporting 
purposes are introduced. The solutions given here represent 
\textit{de~facto} standard, mainstream open-source solutions available 
on the market currently. The focus is on production-quality, mature 
software rather than bleeding-edge development versions of 
\abbrev{OSS}.


% Presentation Tier: Apache Struts
%---------------------------------
\subsection{Presentation Tier: Apache Struts}
\label{toc:oss:selected:struts}

For the presentation tier the Apache Struts framework is described 
\citep{struts}. Struts is a web framework that supports the creation 
of a web presentation tier. From late 2002, Struts became the natural 
choice for Java~\abbrev{EE} web applications \citep{j2eeframeworks}. 
Struts is still the standard choice for the presentation tier in web 
applications, it is widely used and has a large user base 
\citep{advancedwebdevel}. However, it is speculated that this is only 
due to it being the first proper web framework on the market, and that 
more recent frameworks will take its place somewhere in the future.

Apache Struts implements the \abbrev{MVC} design pattern with a slight 
modification to support web usage. In contrast to the traditional 
\abbrev{MVC} model, where the model components \textsl{push} the 
information to the view components, a web \abbrev{MVC} view component 
must \textsl{pull} the information from the model while being rendered 
\citep{j2eeframeworks}. In Struts, the \abbrev{MVC} parts are the 
following \citep{masteringstruts}:

\begin{description}
\item[View] The JavaServer Pages that render the \abbrev{HTML} pages 
shown to user

\item[Controller] The \code{ActionServlet} class, which forwards the 
request to \code{Action} subclasses

\item[Model] Struts does not provide any model classes; these must be 
provided by the application developer
\end{description}

Apache Struts functions around a single front controller, the 
\code{ActionServlet}, which delegates all the requests coming to the 
web application. The controller is configured by a configuration file 
to forward the requests to specific actions, which process them based 
on the input data. The actions can select the resulting page via 
string-keyed mappings -- for example, an action might process the 
request and return a mapping with the key "failure", which in turn can 
be configured to point to a specific view page.

In addition, Struts supports concentrated exception handling by 
creating a subclass of \code{ExceptionHandler} provided by Struts and 
attaching that to the desired actions via the configuration file. This 
allows handling all the exceptions of the entire presentation tier, 
which in turn makes the code of the actual actions simpler.

In Struts, the \abbrev{HTML} input forms that are shown on application 
views are mapped into \code{ActionForm} components, which must define 
the fields of the form as JavaBean properties. Struts automatically 
loads the values sent by the user to the \code{ActionForm} class, 
where they can be easily fetched from. Struts also contains a 
\textsl{Validator} component that can be configured to automatically 
check the input values both on the client side and on the server side. 
The Struts Validator has been later on moved to the Jakarta Commons 
\citep{commons} project, which is a repository of reusable software 
libraries.

The actual \abbrev{(X)HTML} output can be generated with any suitable 
techniques. However, Struts provides supportive tag libraries for 
\abbrev{JSP} that extend the standard \abbrev{JSTL} libraries, and 
supports using the \abbrev{JSTL} Expression Language with Struts EL. 
For example, localised error messages can be rendered to output pages 
with the \abbrev{HTML} tag library function \code{<html:errors~/>}.

The main advantage of Struts is its structure, which is also its main 
weakness. Basically, all Struts applications must be coded in a 
similar fashion. Enforcing a given architecture keeps the code at 
least above a certain level, but on the other hand, it also restricts 
the freedom of the developers.


% Business Tier: Spring Framework
%--------------------------------------
\subsection{Business Tier: Spring Framework}
\label{toc:oss:selected:spring}

The Spring Framework \citep{spring} is a Java~\abbrev{EE} framework 
that is based on code published in \citep{j2eednd}. It was first 
released as an open source project in January 2003, and it has quickly 
become the dominant Java~\abbrev{EE} application framework 
\citep{j2eeframeworks}. While Spring might provide less services than 
full-blown frameworks, \cite{rapidspring} see its lesser complexity as 
a way to increase productivity. Spring provides a way to create 
Java~\abbrev{EE} compliant software without using \abbrev{EJB}.

Spring is a layered framework, which means that developers can choose 
the parts they need in their own application. At the core of Spring is 
the inversion of control container and bean factories, which can be 
used to create and configure the objects required by the application. 
The creation and configuration of these objects, or beans, is based on 
\abbrev{XML} configuration files. All beans under Spring have a 
logical name that is used to setup the object configuration. The beans 
can also be fetched in application code from the Spring container with 
the logical names. In addition to managing references to other beans, 
variables of normal primitive types can also be set via the 
configuration files. This removes the need to have hard-coded 
singleton configuration objects in the application code. Perhaps the 
simplest way to integrate Spring to an application is just to move the 
application object management under Spring configuration. 
\citep{springintro}

However, Spring also provides many other features. For example, the 
\abbrev{JDBC} data sources of an application can be created, 
configured and injected to objects that need them by Spring. Spring 
does not, however, reinvent the wheel. There are no database 
connection classes in Spring, but the standard components of the 
developer's choice can be configured via Spring configuration files. 
Spring supports integration to popular data access technologies, such 
as Hibernate, \abbrev{JDO}, Oracle TopLink and iBATIS.

For basic \abbrev{JDBC} access, Spring provides wrappers that change 
the checked exceptions of \abbrev{JDBC} into unchecked ones. The 
design idea behind this practice is that in most cases an application 
cannot do anything about a database exception. In this situation it is 
useless to force application developers to catch all exceptions. 
\abbrev{JDBC} is one of the few data access \abbrev{API}s that still 
use unchecked exceptions. TopLink, \abbrev{JDO}, and the newest 
version\footnote{Hibernate switched from checked to unchecked 
exceptions in version 3.} of Hibernate use unchecked exceptions 
exclusively. In case of a recoverable exception, developers can still 
catch the unchecked exception, they just are not forced to do so 
\citep{springintro}.

Spring also supports aspect-oriented programming (\abbrev{AOP}), has 
its own web framework (the Spring Web \abbrev{MVC}) and is developed 
to support easy testing. The Spring container can be initialised in a 
single line of code to be used in the entire application as well as in 
unit tests.


% Database Tier: Hibernate
%-------------------------
\subsection{Database Tier: Hibernate}
\label{toc:oss:selected:hibernate}

The database tier requires communication with the actual databases. 
Traditionally this has been accomplished by writing long queries with 
the Structured Query Language (\abbrev{SQL}). Hibernate 
\citep{hibernate} provides an alternative, light-weight approach: 
mapping the model objects into database tables by simple configuration 
files or by annotating the code itself with hints on how to store the 
fields and hierarchies of objects. Hibernate is not the only solution 
for the \abbrev{ORM} issue, but it was the first popular, fully 
featured and open-source solution \citep{j2eeframeworks}, and it has 
grown in popularity.

Hibernate allows developers to write the data objects as Plain Old 
Java Objects, or \abbrev{POJO}s, rather than force the usage of 
\abbrev{EJB} with its remote and local interfaces. The \abbrev{POJO}s 
are mapped either by writing configuration files or by annotating the 
code itself. \textsl{Hibernate Annotations}, which was just released, 
supports using Java 5 annotations to specify the object/relational 
mapping for the persistent objects.

The mapped objects can be stored to and fetched from the database via 
simple methods such as \code{session.get(class, id)} and 
\code{session.saveOrUpdate(object)}. More complex queries can be 
written with Hibernate's replacement for \abbrev{SQL}, the Hibernate 
Query Language. \abbrev{HQL} is similar in syntax with \abbrev{SQL}, 
but references to objects and fields are written with the names of the 
actual classes and their properties instead of the database tables and 
columns. In addition, the \abbrev{HQL} parser obtains information 
about the actual class hierarchy from the mapping configuration, and 
uses that to fill in missing query information. For example, the 
following statement is a perfectly valid \abbrev{HQL} query statement:

\begin{quote}
\code{from User as user where user.company.name = 'Firm'} 
\end{quote}

Object hierarchies are mapped automatically, with no need for 
developers to query them manually. In a full-blown application that 
uses Hibernate it is perfectly normal that there are no \abbrev{SQL} 
queries and only a couple of \abbrev{HQL} ones (mostly for searches). 
All the basic work will be done with the mapping specifications. 
However, Hibernate does allow the usage of standard \abbrev{SQL} too, 
for those situations where \abbrev{HQL} is not enough or a 
non-standard, vendor-specific database extension is required. In fact, 
developers can override all the queries that Hibernate automatically 
constructs with custom \abbrev{SQL} ones.



% Artifact Generation: XDoclet
%-------------------------
\subsection{Artifact Generation: XDoclet}
\label{toc:oss:selected:xdoclet}

For generating source code, the XDoclet engine is presented. XDoclet 
is a widely used, extensible code generation engine 
\citep{aspectizing}. Currently there are two different versions of 
XDoclet, XDoclet~1~\citep{xdoclet1} and XDoclet~2~\citep{xdoclet2}. 
XDoclet~2 is a rewrite of XDoclet~1, and these are developed and 
maintained separately.

XDoclet is an artifact generation engine that supports generating code 
and artifacts for many existing technologies. These include 
\abbrev{EJB}, Hibernate, JBoss, Struts and Spring Framework. The 
artifacts are generated based on XDoclet annotation tags (which 
resemble JavaDoc tags). Usually XDoclet code generation is tied to the 
build process by invoking XDoclet via Ant tasks.

Using XDoclet can reduce the amount of code written dramatically. For 
example, a simple \abbrev{EJB} bean can consist of seven different 
files which need to be maintained. With XDoclet it is possible to 
maintain a single file, from which the rest are generated. 
\citep{xdoclet1}


% IDE: Eclipse
%-------------
\subsection{IDE: Eclipse}
\label{toc:oss:selected:eclipse}

For the development environment, the Eclipse platform \citep{eclipse} 
is introduced. Eclipse has been gaining popularity ever since its 
creation, has a very large support group and is now the Java 
\abbrev{IDE} of choice for the open source developer 
\citep{rapidspring,eclipseide}.

Although Eclipse is actually a rich client platform that supports the 
creation of any software (for example, a BitTorrent client called 
Azureus is developed on top of Eclipse), the Java Development Toolkit 
is the \abbrev{IDE} that is most often associated with the word 
Eclipse.

The current version of Eclipse's Java Development Toolkit provides 
full support for the Java language, including the Java 5 extensions 
(generics, annotations, and so on). The \abbrev{JDT} contains numerous 
time-saving features such as code completion, class import management, 
easy navigation between classes, searching for implementing classes 
and subclasses, automatic JavaDoc showing and multiple refactoring 
tools. Refactoring tools include the generation of getter and setter 
methods for class fields and updating the calling code when renaming 
or moving classes and methods.



% Building: Apache Maven
%-----------------------
\subsection{Building: Apache Maven}
\label{toc:oss:selected:maven}

Any complex project requires a build tool. In traditional C 
development, the Make tool has been the official tool for decades. 
There are, however, shortcomings in Make with cross-platform Java 
development, and so the Ant tool was developed and accepted as a 
substitute for Make in the Java world \citep{ant}. Even though Ant has 
many improvements over Make with cross-platform support and custom 
build tasks, it is still based on the same idea as Make: both are 
configured with long configuration files that contain instructions on 
running all of the required build commands. Apache Maven \citep{maven} 
is a popular open-source build tool based on a different ideology. 
Maven favours the idea of \textsl{conventions over configuration}, 
which means that build settings, such as source code directories and 
build directories, can be shared between projects. In fact, Maven 
defines a standard directory layout for projects. Those projects that 
adhere to the standard layout do not need to configure their 
application paths at all. A new project only needs to initialise the 
Maven directory structure with a single Maven command, and fill in the 
automatically created project object model (\abbrev{POM}) 
configuration file. In the simplest case, only the name of the project 
and the distribution packaging type (for example, \abbrev{JAR} or 
\abbrev{WAR}) needs to be filled in. After that, source code that is 
placed in the correct directory can be automatically built, tested, 
packaged and distributed. \citep{maven,mavenintro}

Maven also introduces a new way of handling project dependencies. 
Required \abbrev{JAR} packages are often kept under version control 
only for distributing them to all project developers. There are no 
real reasons for version controlling the packages themselves. With 
Maven, the packages required by the project are kept in one or 
multiple Maven repositories. The requirements, or dependencies, are 
written to the Maven configuration file. Maven will then automatically 
download the required artifacts from the repositories. Maven provides 
a central repository that contains a number of standard Java project 
requirements, such as logging components and the Apache Commons 
\citep{commons} components. In case the required artifacts are not 
found on the central repository, a company can easily set up its own 
Maven repository. In addition, when running Maven tasks, the 
dependencies are automatically appended to classpath, so there is no 
need for developers to maintain long lists of classpath entries in 
different configuration files. \citep{maven}


% Version Control: Subversion
%----------------------------
\subsection{Version Control: Subversion}
\label{toc:oss:selected:svn}

Any software project that is going to production requires a version 
control to store the different versions of the software and to help 
merge conflicts that happen when two developers modify the same source 
code file at the same time. The dominant open-source version control 
system has been the Concurrent Versions System \cite{cvs}. However, 
\abbrev{CVS} and \abbrev{RCS}, which it is based on, are old 
technologies, and they have not changed to match new software 
development requirements. \citep{versioncontrol} 

A new, compelling open source replacement to \abbrev{CVS} is 
Subversion \citep{svn}. It has been developed to be a replacement of 
\abbrev{CVS}, so developers working with \abbrev{CVS} should have no 
difficulties in moving to Subversion. Most of the problems with 
\abbrev{CVS} have to do with directories, since \abbrev{RCS} does not 
know anything about them. Subversion has been developed to fully 
support directories and to maintain version history when moving files 
around.

Branching in Subversion has also been changed from the traditional 
style. There are no conventional (\abbrev{CVS}-style) branches in 
Subversion; rather, any directory can be virtually copied under a new 
name. The new directory is just a link, and will store only the 
modifications that are made under that directory. The version history 
before the copying point remains the same for both directories. These 
virtual directory copies effectively work as branches, but they can 
also be used for other purposes.

Subversion can be run either as its own server or under the Apache 
\abbrev{HTTP} server. When running under Apache, the Subversion code 
repositories can also be reached via normal \abbrev{HTML} web 
browsers. This allows for easy code and change inspection from outside 
the office.


%%%%%%%%%%%%%%%%%%%%%%%%%%%%%%%%%%%%%%%%%%%%%%%%%%%%%%%%%%%%%%%%%%%%
% Summary
%%%%%%%%%%%%%%%%%%%%%%%%%%%%%%%%%%%%%%%%%%%%%%%%%%%%%%%%%%%%%%%%%%%%
\section{Summary}
\label{toc:oss:summary}

This chapter provided an introduction to open-source software and the 
Open Source Initiative, which maintains the Open Source Definition and 
promotes the use of open-source software in software business industry.

Section~\ref{toc:oss:usage} supplied the motivation and justification 
for using open-source software. While some open source projects can be 
of questionable quality, most of them compete with their proprietary 
counterparts. For some open source projects, technical excellence is 
the primary goal. Using mainstream \abbrev{OSS} can be seen as a safe 
choice.

In section~\ref{toc:oss:selected}, various \abbrev{OS} solutions were 
presented for the problems and requirements identified in 
chapter~\ref{toc:webdevel}. The section listed well-known, mainstream 
\abbrev{OS} solutions that can be taken for production use.


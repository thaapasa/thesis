%%%%%%%%%%%%%%%%%%%%%%%%%%%%%%%%%%%%%%%%%%%%%%%%%%%%%%%%%%%%%%%%%%%%
%-------------------------------------------------------------------
% Project: KAPSELI
%-------------------------------------------------------------------
%%%%%%%%%%%%%%%%%%%%%%%%%%%%%%%%%%%%%%%%%%%%%%%%%%%%%%%%%%%%%%%%%%%%
\selectlanguage{british}
\chapter{Project: KAPSELI}
\label{toc:project}

This chapter describes the 
\finnish{KAPSELI}-project\footnote{\finnish{KAPSELI} is an 
abbreviation of the Finnish words \textsl{\finnish{kiinteist�t, 
asiakkaat, projektit, seuranta, lis�myynti}}.} and its requirements. 
In the project, \finnish{HiQ Softplan Oy} creates a new implementation 
of a web-based real estate transaction system for \finnish{GVA Finland 
Oy}. Crucial attributes of the new system are extensibility, 
maintainability and modularity. The open-source solutions and agile 
software development methods presented in this thesis are applied in 
the execution of the project.

First, the current situation and future needs of \finnish{HiQ Softplan 
Oy}'s client, \finnish{GVA Finland Oy}, are presented. The current 
system is described and the need for a new system is identified. After 
that, the requirements for the new software system are summarised.


%%%%%%%%%%%%%%%%%%%%%%%%%%%%%%%%%%%%%%%%%%%%%%%%%%%%%%%%%%%%%%%%%%%%
% Current Situation
%%%%%%%%%%%%%%%%%%%%%%%%%%%%%%%%%%%%%%%%%%%%%%%%%%%%%%%%%%%%%%%%%%%%
\section{Current Situation}
\label{toc:project:current}

\finnish{GVA Finland Oy} uses an existing web-based real estate 
transaction system. The administrative side of the system is used to 
record information about facilities that are on sale or for rent and 
customers that are selling, renting or purchasing these facilities. 
Other companies can use the public side of the system to search for 
facilities for their own use. Both the administrative side and the 
public side are accessible with a standard web browser.

The current system works fine for this purpose, but \finnish{GVA 
Finland Oy} has new requirements and the system needs to be extended. 
Furthermore, an interactive map service needs to be integrated into 
the system, and the customer information needs to be moved into a 
customer relationship management (\abbrev{CRM}) system. In addition, 
support for generating reports needs to be included in the system.

The new business requirements set more demands for the architecture of 
the system. It was decided to reimplement the system because of the 
new demands and because the original source codes were not available 
at first. The new system is designed to have a more robust structure 
and backing frameworks. The software will be implemented on top of 
widely used open-source solutions using agile software development 
methods.


%%%%%%%%%%%%%%%%%%%%%%%%%%%%%%%%%%%%%%%%%%%%%%%%%%%%%%%%%%%%%%%%%%%%
% Software Requirements
%%%%%%%%%%%%%%%%%%%%%%%%%%%%%%%%%%%%%%%%%%%%%%%%%%%%%%%%%%%%%%%%%%%%
\section{Software Requirements}
\label{toc:project:requirements}

The basic requirements for the first release of the new real estate 
transaction system are threefold. First of all, the new system must 
contain the same functionality than the existing system with the 
exception of customer management, which is moved to a \abbrev{CRM} 
system. Secondly, an interactive map software, which will be provided 
by a third party, needs to be integrated to the system. Finally, the 
customer management side needs to be integrated to a \abbrev{CRM} 
system provided by a third party.

In addition, the software architecture needs to be implemented in such 
a way that it supports extending the basic functionality later on. 
Future extension ideas include custom reports and different types of 
sales projects. Some changes were also planned for the business object
representation and handling in the administrative side of the system. 

It was also decided to renew the layout of the system. In addition to 
creating a new graphical layout to the system, some changes were 
planned for the administrative forms used to manage the real estates 
in the system. The look and feel of these forms was not finalised 
during the planning phase of the project, and it was decided that 
these will be iterated during the development process.
